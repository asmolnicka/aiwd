\documentclass[12pt,a4paper]{article}

\usepackage{polski}
\usepackage[utf8]{inputenc}
\usepackage[T1]{fontenc}
\usepackage{amssymb}
\usepackage{geometry}
\usepackage{xcolor}
\usepackage{hyperref}
\usepackage{float}
\usepackage{graphicx}
\usepackage{caption}
\usepackage{adjustbox}
\usepackage{setspace}

\geometry{margin=2.5cm}
\setstretch{1.2}

\usepackage{titlesec}
\titlelabel{\thetitle.\quad}

\usepackage{enumitem}

\setlist[itemize,1]{label=\tiny$\blacksquare$}   
\setlist[itemize,2]{label=$\circ$}          
\setlist[itemize,3]{label=$\bullet$}        



\definecolor{lightkhaki}{rgb}{0.94, 0.9, 0.55}
\hypersetup{                       
    linktoc=all,
    citebordercolor=lightkhaki,
    linkbordercolor=lightkhaki,
    urlbordercolor=lightkhaki
}


\makeatletter
\newcommand{\thickhline}{%                              
	\noalign {\ifnum 0=`}\fi \hrule height 1pt
	\futurelet \reserved@a \@xhline}

\usepackage{makecell}

\newcommand{\tabelfont}{\fontsize{10}{12}\selectfont}    
\setlength{\extrarowheight}{5pt}                        
\newcolumntype{?}{!{\vrule width 1.2pt}}                



\title{Raportowanie wyników testów\\[1ex]statystycznych dwóch zmiennych}
\author{}
\date{}


\begin{document}

\maketitle
\tableofcontents

\newpage

\section{Struktura raportowania}

\subsection{Rozdział Metody}

\begin{itemize}
    \item \textbf{pytanie badawcze} - pytanie, na które spróbujemy znaleźć odpowiedź
        \begin{itemize}
            \item Bezpośrednio związane z hipotezą badawczą, ale często poprzedzone dodatkowo słowem ,,czy''.
        \end{itemize}

    \item \textbf{hipoteza badawcza} - zakładana odpowiedź na pytanie badawcze
    \begin{itemize}
        \item Od hipotez testowych $H_0$ i $H_1$ różni się tym, że:
        \begin{itemize}
            \item jej sformułowanie jest bardziej ogólne, często dotyczy zmiennych jeszcze sprzed operacjonalizacji, a nie konkretnych zależności w danych,
            \item jest zgodna tylko z jedną z nich - częściej z $H_1$ niż z $H_0$.
        \end{itemize}
    \end{itemize}

    \item \textbf{nazwy zastosowanych testów statystycznego} do zbadania problemu
     \begin{itemize}
        \item Jeśli zmienna jest ilościowa, raportujemy o wykonaniu wybranych testów wstępnych, by sprawdzić założenia odpowiedniego testu parametrycznego, a następnie, na podstawie ich wyników, podajemy ostatecznie wybrany test właściwy. W przypadku pozostałych zmiennych, od razu podajemy nazwę testu właściwego.
        \item Jeśli uzyskany wynik z testu dla więcej niż 2 prób jest istotny statystycznie, to uprzedzamy fakty i podajemy przy nim od razu nazwę testów post-hoc. Jeśli nie jest, pomijamy o nich informację i podajemy jedynie nazwę testu dla kilku prób.
    \end{itemize}   

    \item \textbf{opis wyników testów wstępnych}, jeśli były one wykonane
    \begin{itemize}
        \item Dla testów wstępnych oraz testów post-hoc zwykle nie zamieszczamy w raporcie badanych hipotez.
        \item Wyniki testów wstępnych można zamieścić w tym rozdziale, by w rozdziale ,,Wyniki'' zamieścić już kluczowe i gotowe do interpretacji analizy.
        \item Jeśli z powodu nienormalności rozkładu zmiennej wykonaliśmy skuteczną transformację danych, to należy napisać jakie to było przekształcenie i przedstawić wyniki testu wstępnego jedynie dla już przekształconych danych.
    \end{itemize}

    \item \textbf{ewentualne dodatkowe informacje na temat zastosowania testów}, jeśli przyjęto jakieś założenia
    \begin{itemize}
        \item Jeśli może być coś jeszcze nieoczywistego dla czytelnika, jak wykonano podane testy statystyczne, to należy o tym napisać w raporcie.
    \end{itemize}
\end{itemize}



\subsection{Rozdział Wyniki}

\begin{itemize}
    \item \textbf{hipotezy testowe} - zerowa i alternatywna -\newline
          obie hipotezy, które przeciwstawia sobie wybrany test statystyczny

    \item \textbf{wykres wspólnego rozkładu zmiennych}\newline
          biorących udział w teście statystycznym, wizualizujący badany problem
    \begin{itemize}
        \item \underline{Pod} wykresem powinien znaleźć się podpis rozpoczęty numerem wykresu. Jego postać w zależności od typu wykresu może być na przykład taka:
        \begin{itemize}
            \item wykres bez zmiennej grupującej (wykres rozrzutu, mapa ciepła):\\  
            ,,Rozkład $x$ i $y$ względem siebie''
            \item wykres ze zmienną grupującą:
            \begin{itemize}
                \item ,,Rozkład $x$ względem $y$''
                \item ,,Rozkład $x$ w zależności od $y$''
                \item ,,Rozkład $x$ dla/wśród/w poszczególnych $y$''
                \item ,,Rozkład $x$ poszczególnych $y$''
            \end{itemize}
        \end{itemize}
        \item Jeśli na wykresie pojawiają się skróty lub coś innego wymaga wyjaśnienia, to~należy to dodać do podpisu wykresu.
        \begin{itemize}
            \item W szczególności, jeśli wykres słupkowy zawiera słupki błędów, to należy napisać, co one określają, np. ,,Słupki błędów: odchylenie standardowe''.
        \end{itemize}
        \item Wykres niebędący częścią wykresu panelowego nie powinien zawierać tytułu.
        \item Wykres powinien zawierać podpisy osi i jeśli jest taka potrzeba, odpowiednie dla zmiennych jednostki umieszczone w kwadratowych nawiasach.
        \item Wykres powinien być czytelny, w szczególności nie zawierać zbędnych dla jego zrozumienia elementów.
    \end{itemize}

    \item \textbf{tabele z wynikami testów statystycznych} (właściwego i post-hoc)
    \begin{itemize}
        \item Tabele raportujące wynik testu statystycznego powinny zawierać:
        \begin{itemize}
            \item rozkład badanych zmiennych
            \begin{itemize}
                \item[] Dla testów parametrycznych zwykle warto zawrzeć $n$, $m$, $sd$,
                \item[] a dla nieparametrycznych $n$, $med$, $IQR$.
            \end{itemize}
            \item wartości związane z testem:
            \begin{itemize}
                \item[] stopnie swobody, statystyki testowe, wartości \emph{p}
            \end{itemize}
            \item wielkość efektu (warto nawet jeśli wynik jest statystycznie nieistotny),
        \end{itemize}
        \item \underline{Nad} wykresem powinien znaleźć się podpis rozpoczęty numerem tabeli.
        \item Jeśli na wykresie pojawiają się skróty lub coś innego wymaga wyjaśnienia, to~należy to zawrzeć w adnotacji pod tabelą.
        \begin{itemize}
            \item W szczególności, pod tabelą należy zawrzeć wyjaśnienia oznaczeń stopnia statystycznej istotności wyników - tylko występujących w danej tabeli lub wszystkich używanych w raporcie niezależnie od ich występowania w~danej tabeli.
        \end{itemize}
    \end{itemize}

    \item \textbf{opis wyników testów statystycznych} (właściwego i post-hoc)
    \begin{itemize}
        \item Wnioski z testu statystycznego powinny być zawarte w języku naturalnym, a~odpowiednie statystyki i wartości związane z testem powinny być umieszczone w nawiasach.
        \item Jeśli wyniki przedstawione w zdaniu odnoszą się do zawartości tabeli (lub wykresu), to powinno się zamieścić w nim odpowiednie odwołanie. Jeśli zdania po nim następujące mają to samo odniesienie, nie trzeba już powtarzać tego odwołania.
        \item Jeśli wyniki testu istotności różnic są zamieszczone również w tabeli, to we wniosku przy wyniku nieistotnym statystycznie, można pominąć wielkość efektu, a umieścić go jedynie w tabeli.
    \end{itemize}
\end{itemize}


\subsection{Ogólne wskazania}

\begin{itemize}
    \item Wszystkie liczby w raporcie powinny mieć tą samą liczbę cyfr po przecinku, zazwyczaj wystarczające są dwie.
    \begin{itemize}
        \item Wyjątek w raporcie mogą stanowić wartości $p$, które są na granicy któregoś ze stopni istotności (np. $< 0{,}05$ lub $< 0{,}001$) i dla których przy ustalonej liczbie cyfr może nie być jasne, w którym przedziale się one znajdują — wtedy można podać odpowiednio większą liczbę cyfr.
    \end{itemize}
    \item Znaki dziesiętne w raporcie powinny być dopasowane do języka raportu - w języku polskim znakiem dziesiętnym jest przecinek, a różne liczby możemy w raporcie oddzielać średnikiem. 
    \item Jeśli wartość $p$ wynosi mniej niż $0{,}001$, to we wniosku, opcjonalnie również w tabeli, zamiast dokładnej wartości podajemy ,,$< 0{,}001$''.
    \item Dodanie do testów statystycznych interpretacji wielkości efektu jest wartościowe, ale nie konieczne.
    \item Skróty z pierwszym pojawieniem się w raporcie powinny zostać rozwinięte,\newline
          np. 1. ,,analiza wariancji (ANOVA)'', 2. ,,ANOVA''.
    \item Dozwolone jest umieszczenie w raporcie testu Welcha pod nazwą testu Studenta.
    \item Wyniki testów statystycznych dla wielu prób zależnych raportuje się analogicznie do raportowania wyników testów dla wielu prób niezależnych.
\end{itemize}

\subsection{Uwagi statystyczne}

\begin{itemize}
    \item Przy testach dla dwóch zmiennych obsługa braków danych nie jest konieczna, ponieważ funkcje automatycznie odrzucają wiersze, w których nie ma pary wartości, a nie wiąże się to z utratą żadnej innej pary wartości. Ma ona znaczenie dopiero, gdy zestawiamy ze sobą co najmniej trzy zmienne, np. w wieloczynnikowej analizie wariancji lub w modelu uczenia maszynowego. Niezależnie od liczby zmiennych wymagana jest jednak zamiana konkretnych wartości na braki danych, jeśli za takie można je uznać (np. ,,Nie wiem'' $\rightarrow$ np.NaN).
    \item Testy wstępne wymagane są jedynie dla zmiennych ilościowych.
    \item Testy post-hoc wymagane są jedynie, gdy wynik w teście dla wielu prób jest istotny statystycznie lub opcjonalnie, gdy jest tego bliski.
    \item Testy nieparametryczne, tak samo jak parametryczne, mają swoje założenia. Są one prostsze do spełnienia, ale również przed wykonaniem tych testów należy się upewnić, czy w danym przypadku tak jest, co zwykle można zrobić bez dodatkowych testów. Sprawdzanie tego nie jest jednak wymagane na projekcie.
    \item Zanim wykorzysta się test nieparametryczny dla zmiennej ilościowej, warto spróbować przekształcić dane do rozkładu normalnego za pomocą transformacji do tego służących (np. Boxa-Coxa) i ponownie wykonać test sprawdzający normalność rozkładu. Jeśli okaże się ona skuteczna, można wykonać test parametryczny na przekształconej zmiennej, jednak dla zrozumiałej interpretacji jego wyników statystyki umieszczone we wniosku i tabeli powinny odnosić się do oryginalnej zmiennej. Wizualizacja może być natomiast dla przekształconej zmiennej. Użycie transformacji w przypadku niespełnionych założeń nie jest jednak wymagane na projekcie.
    \item Wielkością efektu dla testu Manna-Whitneya i Wilcoxona może być CL (Common Language) lub $r_{rb}$ - współczynnik korelacji rangowo-dwuseryjnej. Dla testów Studenta dla 2 prób może to być natomiast $d$ Cohena lub $r_{pb}$ - współczynnik korelacji punktowo-dwuseryjnej (przy czym ten drugi do uzyskania wymaga użycia osobnej funkcji).
\end{itemize}





\newpage

%%%%%%%%%%%%%%%%%%%%%%%%%%%%%%%%%%%%%%%%%%%%%%%%%%%%%%%%%%
\section{Testy istotności różnic dla danych niezależnych}
%%%%%%%%%%%%%%%%%%%%%%%%%%%%%%%%%%%%%%%%%%%%%%%%%%%%%%%%%%

\subsection{Dwie grupy}


\subsubsection{Test Welcha (Studenta) dla dwóch prób}

\vspace{0.5cm}

\hrule

\vspace{1cm}

\begin{center}
    \textbf{W rozdziale Metody:}
\end{center}

\vspace{0.5cm}

\noindent
\textbf{Pytanie badawcze}: Czy kobiety i mężczyźni różnią się wskaźnikiem uwagi? \\[0.6ex]
\textbf{Hipoteza badawcza}: Kobiety i mężczyźni różnią się wskaźnikiem uwagi.

\vspace{0.5cm}

\noindent
\textbf{Metoda sprawdzająca założenie} testu Welcha/Studenta - normalność rozkładu:

\vspace{0.2cm}

\noindent
test Shapiro-Wilka

\vspace{0.5cm}

\noindent
\textbf{Wynik testów wstępnych}:

\vspace{0.2cm}

\noindent
Na poziomie istotności 0,05 nie mamy podstaw do odrzucenia hipotezy zerowej ( $W~=~1{,}00$~; $p = 0{,}89$ ), a więc przyjmiemy, że wskaźnik uwagi wśród grup poszczególnych płci wystarczająco spełnia założenie normalności w teście Studenta.

\vspace{0.5cm}

\textbf{Wynik testów wstępnych}, gdyby wynik był istotny statystycznie:

\vspace{0.2cm}

Wskaźnik uwagi wśród grup poszczególnych płci nie ma rozkładu normalnego\\
\indent ( $ W = 0{,}94$ ; $p < 0{,}001$ ).

\vspace{0.5cm}

\noindent
\textbf{Metoda}: test Welcha/Studenta dla dwóch prób niezależnych

\vspace{1cm}

\hrule

\vspace{0.5cm}

\begin{center}
    \textbf{W rozdziale Wyniki:}
\end{center}

\vspace{0.5cm}

[wykres rozkładu łącznego zmiennych]

\vspace{0.5cm}

\noindent
\textbf{Hipotezy testowe}:

\vspace{0.2cm}

\noindent
$H_0:$ Średni wskaźnik uwagi u kobiet jest równy średniemu wskaźnikowi uwagi u mężczyzn. \\[0.6ex]
$H_1:$ Średni wskaźnik uwagi u kobiet jest różny od średniego wskaźnika uwagi u mężczyzn.

\vspace{0.5cm}

\begin{table}[H]
\centering
\caption{Porównanie wskaźnika uwagi między płciami}
\label{tab:attention_gender}
\vspace{0.35cm}
\tabelfont{
\begin{adjustbox}{center}
\begin{tabular}{l?ccc?cccc?}
    \cline{2-8}
     & \multicolumn{3}{c?}{\textbf{wskaźnik uwagi}} & \multicolumn{4}{c?}{\textbf{test Welcha/Studenta}}    \\[0.6ex]
    \hline
\multicolumn{1}{|c?}{\textbf{płeć}} & $n$ & $m$          & $sd$         & $df$     & $T$     & $p$  & $d$   \\[0.6ex]
    \thickhline
\multicolumn{1}{|c?}{kobieta}   & 116 & 0,59             & 0,14         & 239,17   & -1,03   & 0,30  & 0,12 \\
\multicolumn{1}{|c?}{mężczyzna} & 184 & 0,60             & 0,14         &          &         &       &      \\[0.6ex]
    \hline
\end{tabular}
\end{adjustbox}}
%{\raggedleft \scriptsize \hphantom{.} \\ $\ast - p \leqslant 0{,}05$; \ $\ast \ast - p \leqslant 0{,}01$; \ $\ast \ast \ast - p \leqslant 0{,}001$}
\end{table}



\vspace{0.5cm}

\noindent
\textbf{Wniosek}:

\vspace{0.2cm}

\noindent
Na poziomie istotności 0,05 nie możemy stwierdzić, czy wskaźnik uwagi u kobiet ( $m~=~0{,}59$~; $sd = 0{,}14$ ) jest różny od wskaźnika uwagi u mężczyzn ( $m = 0{,}60$ ; $sd = 0{,}14$ ) (~$T(239{,}17) = -1{,}03$ ; $p = 0{,}30$ ) (Tab. \ref{tab:attention_gender}).  % $r_{pb} = -0{,}06$

\vspace{0.5cm}

\noindent
\textbf{Wniosek}, gdyby wynik był istotny statystycznie:

\vspace{0.2cm}

\noindent
Wskaźnik uwagi u kobiet ( $m = ...$ ; $sd = ...$ ) jest różny od wskaźnika uwagi u mężczyzn ( $m = ...$ ; $sd = ...$ ) ( $T(...) = ...$ ; $p = ...$ ; $d = ...$ ) (Tab. \ref{tab:attention_gender}).

\vspace{1cm}

\hrule

\vspace{0.5cm}

\subsubsection{Jednostronny test Welcha (Studenta) dla dwóch prób}

\vspace{0.5cm}

\hrule

\vspace{0.5cm}

\textbf{Pytanie badawcze}: Czy kobiety i mężczyźni różnią się wskaźnikiem uwagi? \\[0.6ex]
\textbf{Hipoteza badawcza}: Kobiety mają większy wskaźnik uwagi niż mężczyźni.

\vspace{0.5cm}

\noindent
\textbf{Metoda}: jednostronny test Welcha/Studenta dla dwóch prób niezależnych

\vspace{0.5cm}

\noindent
\textbf{Hipotezy testowe}:

\vspace{0.2cm}

\noindent
$H_0:$ Średni wskaźnik uwagi u kobiet jest równy średniemu wskaźnikowi uwagi u mężczyzn. \\[0.6ex]
$H_1:$ Średni wskaźnik uwagi u kobiet jest większy od średniego wskaźnika uwagi u mężczyzn.

\vspace{0.5cm}

\noindent
\textbf{Wniosek}:

\vspace{0.2cm}

\noindent
Na poziomie istotności 0,05 nie możemy stwierdzić, czy wskaźnik uwagi u kobiet ( $m~=~0{,}59$~; $sd = 0{,}14$ ) jest większy od wskaźnika uwagi u mężczyzn ( $m = 0{,}60$ ; $sd = 0{,}14$ ) (~$T(239{,}17) = -1{,}03$ ; $p = 0{,}85$ ).

\vspace{0.5cm}

[pozostałe punkty analogicznie do dwustronnego testu Welcha]

\vspace{1cm}

\hrule

\vspace{0.5cm}

\subsubsection{Test Manna–Whitneya}

\vspace{0.5cm}

\hrule

\vspace{1cm}

\begin{center}
    \textbf{W rozdziale Metody:}
\end{center}

\vspace{0.5cm}

\noindent
\textbf{Pytanie badawcze}: Czy kobiety i mężczyźni różnią się poziomem wykształcenia? \\[0.6ex]
\textbf{Hipoteza badawcza}: Kobiety i mężczyźni różnią się poziomem wykształcenia.

\vspace{0.5cm}

[test wstępny, jeśli zmienna jest ilościowa]

\vspace{0.5cm}

\noindent
\textbf{Metoda}: test Manna-Whitneya

\vspace{1cm}

\hrule

\vspace{0.5cm}

\begin{center}
    \textbf{W rozdziale Wyniki:}
\end{center}

\vspace{0.5cm}

[wykres rozkładu łącznego zmiennych]

\vspace{0.5cm}

\noindent
\textbf{Hipotezy testowe}:

\vspace{0.2cm}

\noindent
$H_0:$ Mediana poziomu wykształcenia u kobiet jest równa medianie poziomu wykształcenia u mężczyzn. \\[0.6ex]
$H_1:$ Mediana poziomu wykształcenia u kobiet jest różna od mediany poziomu wykształcenia u mężczyzn.

\vspace{0.5cm}

\begin{table}[H]
\centering
\caption{Porównanie poziomu wykształcenia między płciami}
\label{tab:education_gender}
\vspace{0.35cm}
\tabelfont{
\begin{adjustbox}{center}
\begin{tabular}{l?>{\hspace{1.5em}}ccc<{\hspace{1.5em}}?>{\hspace{1.7em}}ccc<{\hspace{1.7em}}?}
    \cline{2-7}
      & \multicolumn{3}{c?}{\textbf{poziom wykształcenia}} & \multicolumn{3}{c?}{\textbf{test Manna–Whitneya}} \\[0.6ex]
    \hline
\multicolumn{1}{|c?}{\textbf{płeć}} & $n$ & $med$        & $IQR$             & $U$     & $p$     & $CL$ \\[0.6ex]
    \thickhline
\multicolumn{1}{|c?}{kobieta}       & 116 & 2            & 2                 & 190     & 0,02    & 0,31 \\
\multicolumn{1}{|c?}{mężczyzna}     & 184 & 3            & 2                 &         & $\ast$  &      \\[0.6ex]
    \hline
\end{tabular}
\end{adjustbox}}
{\raggedleft \scriptsize \hphantom{.} \\
$\ast - p \leqslant 0{,}05$
%;\ $\ast \ast - p \leqslant 0{,}01$;\ $\ast \ast \ast - p \leqslant 0{,}001$
}
\end{table}


\vspace{0.5cm}

\noindent
\textbf{Wniosek}:

\vspace{0.2cm}

\noindent
Kobiety różnią się poziomem wykształcenia ( $med = 2$ ; $IQR = 2$ ) od mężczyzn (~$med~=~3$~; $IQR = 2$ ) ( $U = 190$ ; $p = 0{,}02$ ; $CL = 0{,}31$ ) (Tab. \ref{tab:education_gender}).  % $r_{bc} = -0{,}38$

\vspace{0.5cm}

\noindent
\textbf{Wniosek}, gdyby wynik nie był istotny statystycznie:

\vspace{0.2cm}

\noindent
Na poziomie istotności 0,05 nie możemy stwierdzić, czy kobiety różnią się poziomem wykształcenia ( $med = ...$ ; $IQR = ...$ ) od mężczyzn ( $med = ...$ ; $IQR = ...$ ) ( $U = ...$ ; $p = ...$ ) (Tab. \ref{tab:education_gender}).

\vspace{1cm}

\hrule

\vspace{0.5cm}



\subsection{Więcej niż dwie grupy}

\subsubsection{Test jednoczynnikowa ANOVA}

\vspace{0.5cm}

\hrule

\vspace{1cm}

\begin{center}
    \textbf{W rozdziale Metody:}
\end{center}

\vspace{0.5cm}

\noindent
\textbf{Pytanie badawcze}:\\
Czy tematy treści różnią się zmiennością nastroju jaka następuje przy interakcji z nimi? \\[0.6ex]
\textbf{Hipoteza badawcza}:\\
Tematy treści różnią się zmiennością nastroju jaka następuje przy interakcji z nimi.

\vspace{0.5cm}

\noindent
\textbf{Metoda sprawdzająca założenie} testu ANOVA - normalność rozkładu:

\vspace{0.2cm}

\noindent
test Shapiro-Wilka

\vspace{0.5cm}

\noindent
\textbf{Metoda sprawdzająca założenie} testu ANOVA - homogeniczność wariancji:

\vspace{0.2cm}

\noindent
test Levene'a

\vspace{0.5cm}

\noindent
\textbf{Wynik testów wstępnych}:

\vspace{0.2cm}

\noindent
Na poziomie istotności 0,05 nie mamy podstaw do odrzucenia hipotezy zerowej (~$W~=~1{,}00$~; $p = 0{,}55$ ), a więc przyjmiemy, że wskaźnik zmienności nastroju w trakcie interakcji z~treściami wśród poszczególnych ich tematów wystarczająco spełnia założenie normalności w~teście ANOVA.

\vspace{0.2cm}

\noindent
Na poziomie istotności 0,05 nie mamy podstaw do odrzucenia hipotezy zerowej (~$W~=~0{,}31$~; $p = 0{,}82$ ), a więc przyjmiemy, że wskaźnik zmienności nastroju w trakcie interakcji z~treściami wśród poszczególnych ich tematów wystarczająco spełnia założenie homogeniczności wariancji w~teście ANOVA.

\vspace{0.5cm}

\noindent
\textbf{Metoda}: jednoczynnikowa analiza wariancji (ANOVA) z testami post-hoc HSD Tukeya\\
\hphantom{.} \hspace{1.45cm} [o testach post-hoc wspomnieć, o ile ANOVA jest istotna]

\vspace{1cm}

\hrule

\vspace{0.5cm}

\begin{center}
    \textbf{W rozdziale Wyniki:}
\end{center}

\vspace{0.5cm}

[wykres rozkładu łącznego zmiennych]

\vspace{0.5cm}

\noindent
\textbf{Hipotezy testowe}:

\vspace{0.2cm}

\noindent
$H_0:$ Średni wskaźnik zmienności nastroju w trakcie interakcji z treściami jest taka sama dla wszystkich tematów tych treści. \\[0.6ex]
$H_1:$ Średni wskaźnik zmienności nastroju w trakcie interakcji z treściami różni się w co najmniej jednej parze tematów tych treści.

\vspace{0.5cm}

\begin{table}[H]
\centering
\caption{Jednoczynnikowa ANOVA dla wskaźnika zmienności nastroju w zależności od tematu treści}
\label{tab:mood_topic}
\vspace{0.35cm}
\tabelfont{
\begin{adjustbox}{center}
\begin{tabular}{|l?ccc?ccc|}
    \hline
    źródło zmienności & $df$ & $SS$ & $MS$ & $F$  & $p$    & $\eta^2$ \\[0.6ex]
    \thickhline
    między grupami    & 3    & 0,24 & 0,08 & 3,59 & 0,014   & 0,04     \\
    wewnątrz grup     & 296  & 6,73 & 0,02 &      & $\ast$ &         \\[0.6ex]
    \hline
\end{tabular}
\end{adjustbox}}
{\raggedleft \scriptsize \hphantom{.} \\
$\ast - p \leqslant 0{,}05$
%; \ $\ast \ast - p \leqslant 0{,}01$; \ $\ast \ast \ast - p \leqslant 0{,}001$
}
\end{table}


\vspace{0.5cm}

\noindent
\textbf{Wniosek z testu ANOVA}:

\vspace{0.2cm}

\noindent
Tematy treści różnią się zmiennością nastroju jaka następuje przy interakcji z nimi\\
( $F(3; 296) = 3{,}59$ ; $p = 0{,}014$ ; $\eta^2 = 0{,}04$ ) (Tab. \ref{tab:mood_topic}).

\vspace{0.5cm}

\textbf{Wniosek}, gdyby wynik nie był istotny statystycznie:

\vspace{0.2cm}

Na poziomie istotności 0,05 nie można stwierdzić, czy tematy treści różnią się \\
\indent zmiennością nastroju jaka następuje przy interakcji z nimi ( $F(...) = ...$ ; $p = ...$ )\\
\indent (Tab. \ref{tab:mood_topic}).

\vspace{1cm}

\begin{table}[H]
\centering
\caption{Porównania parami dla wskaźnika zmienności nastroju w zależności od tematu treści}
\label{tab:mood_topic_ph}
\vspace{0.35cm}
\tabelfont{
\begin{adjustbox}{center}
\begin{tabular}{l?>{\hspace{1.5em}}ccc<{\hspace{1.5em}}?cccc?}
    \cline{2-8}
& \multicolumn{3}{c?}{\textbf{zmienność nastroju}} & \multicolumn{4}{c?}{\textbf{test HSD Tukeya}} \\[0.6ex]
    \hline
\multicolumn{1}{|c?}{\textbf{temat}}    & $n$ & $m$ & $sd$ & $df$ & $T$  & $p$    & $g$     \\[0.6ex]
    \thickhline
\multicolumn{1}{|c?}{radzenie sobie}    & 84 & 0,54 & 0,16 & 296 & 2,70  & 0,04   & 0,41    \\
\multicolumn{1}{|c?}{uważność}          & 83 & 0,47 & 0,15 &     &       & $\ast$ &         \\[0.6ex]
    \hline
\multicolumn{1}{|c?}{radzenie sobie}    & 84 & 0,54 & 0,16 & 296 & -0,02 & 1,00   & -0,00   \\
\multicolumn{1}{|c?}{budowa odporności} & 64 & 0,54 & 0,14 &     &       &        &         \\[0.6ex]
    \hline
\multicolumn{1}{|c?}{\vdots} & \vdots & \vdots & \vdots & \vdots & \vdots & \vdots & \vdots \\[0.6ex] 
\end{tabular}
\end{adjustbox}}
\vspace{0.5cm}
{\raggedleft \scriptsize \hphantom{.} \\
$\ast - p \leqslant 0{,}05$; \ $\ast \ast - p \leqslant 0{,}01$; \ $\ast \ast \ast - p \leqslant 0{,}001$}
\end{table}



\vspace{0.5cm}

\noindent
\textbf{Wnioski z testów post-hoc}:

\vspace{0.2cm}

\noindent
Temat umiejętności radzenia sobie różni się zmiennością nastroju jaka następuje przy interakcji z nimi ( $m = 0{,}54$~; $sd = 0{,}16$ ) od tematu uważności ( $m = 0{,}47$ ; $sd = 0{,}15$ ) (~$T(296) = 2{,}70$ ; $p = 0{,}04$ ; $g = 0{,}41$ ) (Tab. \ref{tab:mood_topic_ph}).

\vspace{0.35cm}

\noindent
Temat uważności różni się zmiennością nastroju jaka następuje przy interakcji z nimi od tematu zarządzania stresem ( $m = 0{,}54$~; $sd = 0{,}15$ ) ( $T(296) = -2{,}63$ ; $p = 0{,}04$ ; $g = -0{,}43$ ).

\vspace{0.35cm}

\noindent
Dla pozostałych par tematów, na poziomie istotności 0,05, nie można stwierdzić różnicy w zmienności nastroju jaka następuje przy interakcji z nimi.


\vspace{1cm}

\hrule

\vspace{0.5cm}


\subsubsection{Test jednoczynnikowa ANOVA Welcha}

\vspace{0.5cm}

\hrule

\vspace{1cm}

\begin{center}
    \textbf{W rozdziale Metody:}
\end{center}

\vspace{0.5cm}

\noindent
\textbf{Pytanie badawcze}:\\ Czy tematy treści różnią się zmiennością nastroju jaka następuje przy interakcji z nimi? \\[0.6ex]
\textbf{Hipoteza badawcza}:\\ Tematy treści różnią się zmiennością nastroju jaka następuje przy interakcji z nimi.

\vspace{0.5cm}
\newpage
\noindent
\textbf{Metoda sprawdzająca założenie} testu ANOVA Welcha - normalność rozkładu:

\vspace{0.2cm}

\noindent
test Shapiro-Wilka

\vspace{0.5cm}

\noindent
\textbf{Wynik testów wstępnych}:

\vspace{0.2cm}

\noindent
Na poziomie istotności 0,05 nie mamy podstaw do odrzucenia hipotezy zerowej (~$W~=~1{,}00$~; $p = 0{,}55$ ), a więc przyjmiemy, że wskaźnik zmienności nastroju w trakcie interakcji z~treściami wśród poszczególnych ich tematów wystarczająco spełnia założenie normalności w~teście ANOVA Welcha.

\vspace{0.5cm}

\noindent
\textbf{Metoda}: jednoczynnikowa analiza wariancji (ANOVA) Welcha z testami post-hoc Gamesa-Howella\\
\hphantom{.} \hspace{1.45cm} [o testach post-hoc wspomnieć, o ile ANOVA jest istotna]

\vspace{1cm}

\hrule

\vspace{0.5cm}

\begin{center}
    \textbf{W rozdziale Wyniki:}
\end{center}

\vspace{0.5cm}

[wykres rozkładu łącznego zmiennych]

\vspace{0.5cm}

\noindent
\textbf{Hipotezy testowe}:

\vspace{0.2cm}

\noindent
$H_0:$ Średni wskaźnik zmienności nastroju w trakcie interakcji z treściami jest taka sama dla wszystkich tematów tych treści. \\[0.6ex]
$H_1:$ Średni wskaźnik zmienności nastroju w trakcie interakcji z treściami różni się w co najmniej jednej parze tematów tych treści.

\vspace{0.5cm}

\begin{table}[H]
\centering
\caption{Jednoczynnikowa ANOVA Welcha dla wskaźnika zmienności nastroju w zależności od tematu treści}
\label{tab:mood_topic_welch}
\vspace{0.35cm}
\tabelfont{
\begin{adjustbox}{center}
\begin{tabular}{|cccc<{\hspace{-0.65em}}lc|}
    \hline
    $df_1$ & $df_2$ & $F$  & $p$   &        & $\eta^2$ \\[0.6ex]
    \thickhline
    3      & 161,72 & 3,73 & 0,012 & $\ast$ & 0,04     \\[0.6ex]
    \hline
\end{tabular}
\end{adjustbox}}
{\raggedleft \scriptsize \hphantom{.} \\
$\ast - p \leqslant 0{,}05$
%; \ $\ast \ast - p \leqslant 0{,}01$; \ $\ast \ast \ast - p \leqslant 0{,}001$
}
\end{table}


\vspace{0.5cm}

\noindent
\textbf{Wniosek z testu ANOVA}:

\vspace{0.2cm}

\noindent
Tematy treści różnią się zmiennością nastroju jaka następuje przy interakcji z nimi\\
( $F(3; 161{,}72) = 3{,}73$ ; $p = 0{,}012$ ; $\eta^2 = 0{,}04$ ) (Tab. \ref{tab:mood_topic_welch}).

\vspace{0.5cm}
\newpage
\textbf{Wniosek}, gdyby wynik nie był istotny statystycznie:

\vspace{0.2cm}

Na poziomie istotności 0,05 nie można stwierdzić, czy tematy treści różnią się \\
\indent zmiennością nastroju jaka następuje przy interakcji z nimi ( $F(...) = ...$ ; $p = ...$ )\\
\indent (Tab. \ref{tab:mood_topic_welch}).

\vspace{1cm}

\begin{table}[H]
\centering
\caption{Porównania parami dla wskaźnika zmienności nastroju w zależności od tematu treści}
\label{tab:mood_topic_welch_ph}
\vspace{0.35cm}
\tabelfont{
\begin{adjustbox}{center}
\begin{tabular}{l?>{\hspace{1.5em}}ccc<{\hspace{1.5em}}?cccc?}
    \cline{2-8}
& \multicolumn{3}{c?}{\textbf{zmienność nastroju}} & \multicolumn{4}{c?}{\textbf{test Gamesa-Howella}} \\[0.6ex]
    \hline
\multicolumn{1}{|c?}{\textbf{temat}}    & $n$ & $m$ & $sd$ & $df$   & $T$   & $p$    & $g$   \\[0.6ex]
    \thickhline
\multicolumn{1}{|c?}{radzenie sobie}    & 81 & 0,54 & 0,16 & 159,51 & 2,66  & 0,04   & 0,41  \\
\multicolumn{1}{|c?}{uważność}          & 83 & 0,47 & 0,15 &        &       & $\ast$ &       \\[0.6ex]
    \hline
\multicolumn{1}{|c?}{radzenie sobie}    & 81 & 0,54 & 0,16 & 140,80 & -0,02 & 1,00   & -0,00 \\
\multicolumn{1}{|c?}{budowa odporności} & 83 & 0,54 & 0,14 &        &       &        &       \\[0.6ex]
    \hline
\multicolumn{1}{|c?}{\vdots} & \vdots & \vdots & \vdots & \vdots & \vdots & \vdots & \vdots \\[0.6ex] 
\end{tabular}
\end{adjustbox}}
\vspace{0.5cm}
{\raggedleft \scriptsize \hphantom{.} \\
$\ast - p \leqslant 0{,}05$; \ $\ast \ast - p \leqslant 0{,}01$; \ $\ast \ast \ast - p \leqslant 0{,}001$}
\end{table}




\vspace{0.5cm}

\noindent
\textbf{Wnioski z testów post-hoc}:

\vspace{0.2cm}

\noindent
Temat umiejętności radzenia sobie różni się zmiennością nastroju jaka następuje przy interakcji z nimi ( $m = 0{,}54$~; $sd = 0{,}16$ ) od tematu uważności ( $m = 0{,}47$ ; $sd = 0{,}15$ ) (~$T(159{,}51) = 2{,}66$ ; $p = 0{,}04$ ; $g = 0{,}41$ ) (Tab. \ref{tab:mood_topic_ph}).

\vspace{0.35cm}

\noindent
Temat uważności różni się zmiennością nastroju jaka następuje przy interakcji z nimi od tematu budowania odporności ( $m = 0{,}54$ ; $sd = 0{,}15$ ) (~$T(136{,}40) = -2{,}67$ ; $p = 0{,}04$~; $g = -0{,}44$ ).

\vspace{0.35cm}

\noindent
Temat uważności różni się zmiennością nastroju jaka następuje przy interakcji z nimi od tematu zarządzania stresem ( $m = 0{,}54$~; $sd = 0{,}15$ ) ( $T(147{,}90) = -2{,}67$ ; $p = 0{,}04$ ; $g = -0{,}43$ ).

\vspace{0.35cm}

\noindent
Dla pozostałych par tematów, na poziomie istotności 0,05, nie można stwierdzić różnicy w zmienności nastroju jaka następuje przy interakcji z nimi.


\vspace{1cm}

\hrule

\vspace{0.5cm}


\subsubsection{Test Kruskala–Wallisa}

\vspace{0.5cm}

\hrule

\vspace{1cm}

\begin{center}
    \textbf{W rozdziale Metody:}
\end{center}

\vspace{0.5cm}

\noindent
\textbf{Pytanie badawcze}: Czy typy treści różnią się poziomem trudności? \\[0.6ex]
\textbf{Hipoteza badawcza}: Typy treści różnią się poziomem trudności.

\vspace{0.5cm}

[test wstępny, jeśli zmienna jest ilościowa]

\vspace{0.5cm}

\noindent
\textbf{Metoda}: test Kruskala-Wallisa z testami post-hoc Dunn\\
\hphantom{.} \hspace{1.45cm} [o testach post-hoc wspomnieć, o ile test KW jest istotny]

\vspace{1cm}

\hrule

\vspace{0.5cm}

\begin{center}
    \textbf{W rozdziale Wyniki:}
\end{center}

\vspace{0.5cm}

[wykres rozkładu łącznego zmiennych]

\vspace{0.5cm}

\noindent
\textbf{Hipotezy testowe}:

\vspace{0.2cm}

\noindent
$H_0:$ Mediana poziomu trudności jest taka sama dla wszystkich typów treści. \\[0.6ex]
$H_1:$ Mediana poziomu trudności różni się w co najmniej jednej parze typów treści.   

\vspace{0.5cm}

\begin{table}[H]
\centering
\caption{Test Kruskala-Wallisa dla poziomu trudności w zależności od typu treści}
\label{tab:difficulty_type}
\vspace{0.35cm}
\tabelfont{
\begin{adjustbox}{center}
\begin{tabular}{|ccc<{\hspace{-0.65em}}lc|}
    \hline
    $df$ & $H$   & $p$  &                & $\eta^2$ \\[0.6ex]
    \thickhline
    3    & 25,71 & 0,00 & $\ast\ast\ast$ & 0,09     \\[0.6ex]
    \hline
\end{tabular}
\end{adjustbox}}
{\raggedleft \scriptsize \hphantom{.} \\
%$\ast - p \leqslant 0{,}05$; \ $\ast \ast - p \leqslant 0{,}01$; \
$\ast \ast \ast - p \leqslant 0{,}001$}
\end{table}


\vspace{0.5cm}

\noindent
\textbf{Wniosek z testu Kruskala-Wallisa}:

\vspace{0.2cm}

\noindent
Typy treści różnią się poziomem trudności ( $H(3) = 25{,}71$ ; $p < 0{,}001$ ; $\eta^2 = 0{,}09$ ) (Tab.~\ref{tab:difficulty_type}).

\vspace{0.5cm}

\textbf{Wniosek}, gdyby wynik nie był istotny statystycznie:

\vspace{0.2cm}

Na poziomie istotności 0,05 nie można stwierdzić, czy typy treści różnią się poziomem\\ \indent trudności ( $H(...) = ...$ ; $p = ...$ )  (Tab. \ref{tab:difficulty_type}).

\vspace{1cm}

\begin{table}[H]
\centering
\caption{Porównania parami dla poziomu trudności w zależności od typu treści}
\label{tab:difficulty_type_ph}
\vspace{0.35cm}
\tabelfont{
\begin{adjustbox}{center}
\begin{tabular}{l?>{\hspace{1.5em}}ccc<{\hspace{1.5em}}?c?}
    \cline{2-5}
& \multicolumn{3}{c?}{\textbf{poziom trudności}} & \multicolumn{1}{c?}{\textbf{test Dunn}} \\[0.6ex]
    \hline
\multicolumn{1}{|c?}{\textbf{treść}} & $n$ & $med$ & $IQR$ & $p$    \\[0.6ex]
    \thickhline
\multicolumn{1}{|c?}{artykuł}        & 95  & 1 & 1 & 0,11           \\
\multicolumn{1}{|c?}{interaktywna}   & 110 & 2 & 2 &                \\[0.6ex]
    \hline
\multicolumn{1}{|c?}{artykuł}        & 95  & 1 & 1 & 0,00           \\
\multicolumn{1}{|c?}{quiz}           & 12  & 3 & 0 & $\ast\ast\ast$ \\[0.6ex]
    \hline
\multicolumn{1}{|c?}{\vdots}         & \vdots & \vdots & \vdots & \vdots \\[0.6ex] 
\end{tabular}
\end{adjustbox}}
\vspace{0.5cm}
{\raggedleft \scriptsize \hphantom{.} \\
$\ast - p \leqslant 0{,}05$; \ $\ast \ast - p \leqslant 0{,}01$; \ $\ast \ast \ast - p \leqslant 0{,}001$}
\end{table}

\vspace{0.5cm}

\noindent
\textbf{Wnioski z testów post-hoc}:

\vspace{0.2cm}

\noindent
Treść typu quiz różni się poziomem trudności ( $med = 3$ ; $IQR = 0$ ) od treści typu artykuł ( $med = 1$ ; $IQR = 1$ ) ( $p < 0{,}001$ ) (Tab. \ref{tab:difficulty_type_ph}).

\vspace{0.35cm}

\noindent
Treść typu quiz różni się poziomem trudności od treści typu interaktywnego ( $med = 2$ ; $IQR = 2$ ) ( $p < 0{,}001$ ).

\vspace{0.35cm}

\noindent
Treść typu quiz różni się poziomem trudności od treści typu wideo ( $med = 1$ ; $IQR = 2$~) ( $p < 0{,}001$ ).

\vspace{0.35cm}

\noindent
Dla pozostałych par typów treści, na poziomie istotności 0,05, nie można stwierdzić różnicy w poziomie trudności.


\vspace{1cm}

\hrule

\vspace{0.5cm}


%%%%%%%%%%%%%%%%%%%%%%%%%%%%%%%%%%%%%%%%%%%%%%%%%%%%%%%
\section{Testy istotności różnic dla danych zależnych}
%%%%%%%%%%%%%%%%%%%%%%%%%%%%%%%%%%%%%%%%%%%%%%%%%%%%%%%

\subsection{Dwie grupy}

\subsubsection{Test Studenta dla dwóch prób zależnych}

\vspace{0.5cm}

\hrule

\vspace{1cm}

\begin{center}
    \textbf{W rozdziale Metody:}
\end{center}

\vspace{0.5cm}

\noindent
\textbf{Pytanie badawcze}:\\
Czy pomiary stresu przed i po interakcji użytkownika z treścią różnią się od siebie? \\[0.6ex]
\textbf{Hipoteza badawcza}:\\
Pomiary stresu przed i po interakcji użytkownika z treścią różnią się od siebie.

\vspace{0.5cm}

\noindent
\textbf{Metoda sprawdzająca założenie} testu Studenta - normalność rozkładu:

\vspace{0.2cm}

\noindent
test Shapiro-Wilka

\vspace{0.5cm}

\noindent
\textbf{Wynik testów wstępnych}:

\vspace{0.2cm}

\noindent
Na poziomie istotności 0,05 nie mamy podstaw do odrzucenia hipotezy zerowej (~$W~=~1{,}00$~; $p = 0{,}89$ ), a więc przyjmiemy, że różnica między pomiarami stresu przed interakcją użytkownika z treścią i po tej interakcji wystarczająco spełnia założenie normalności w teście Studenta.

\vspace{0.5cm}

\textbf{Wynik testów wstępnych}, gdyby wynik był istotny statystycznie:

\vspace{0.2cm}

Różnica między pomiarami stresu przed interakcją użytkownika z treścią i po tej\\
\indent interakcji nie ma rozkładu normalnego ( $W = 0{,}94$ ; $p < 0{,}001$ ).

\vspace{0.5cm}

\noindent
\textbf{Metoda}: test Studenta dla dwóch prób zależnych

\vspace{1cm}

\hrule

\vspace{0.5cm}

\begin{center}
    \textbf{W rozdziale Wyniki:}
\end{center}

\vspace{0.5cm}

[wykres rozkładu łącznego zmiennych]

\vspace{0.5cm}

\noindent
\textbf{Hipotezy testowe}:

\vspace{0.2cm}

\noindent
$H_0:$ Średnia różnica między pomiarami stresu przed i po interakcji użytkownika z treścią wynosi 0. \\[0.6ex]
$H_1:$ Średnia różnica między pomiarami stresu przed i po interakcji użytkownika z treścią jest różna od 0.

\vspace{0.5cm}

\begin{table}[H]
\centering
\caption{Porównanie pomiarów stresu przed i po interakcji}
\label{tab:stress_12_t}
\vspace{0.35cm}
\tabelfont{
\begin{adjustbox}{center}
\begin{tabular}{l?ccc?cccc?}
    \cline{2-8}
     & \multicolumn{3}{c?}{\textbf{stres}} & \multicolumn{4}{c?}{\textbf{test Studenta}}     \\[0.6ex]
    \hline
\multicolumn{1}{|c?}{\textbf{pomiar}}  & $n$ & $m$ & $sd$ & $df$ & $T$ & $p$ & $d$ \\[0.6ex]
    \thickhline
\multicolumn{1}{|c?}{przed interakcją} & 300 & 25,13 & 5,95 & 299 & 45,62 & 0,00 & 0,52 \\
\multicolumn{1}{|c?}{po interakcji}    & 300 & 22,02 & 5,93 & & & $\ast\ast\ast$ & \\[0.6ex]
    \hline
\end{tabular}
\end{adjustbox}}
{\raggedleft \scriptsize \hphantom{.} \\
%$\ast - p \leqslant 0{,}05$; \ $\ast \ast - p \leqslant 0{,}01$; \
$\ast \ast \ast - p \leqslant 0{,}001$}
\end{table}


\vspace{0.5cm}
\newpage
\noindent
\textbf{Wniosek}:

\vspace{0.2cm}

\noindent
Pomiary stresu przed interakcją użytkownika z treścią ( $m = 25{,}13$ ; $sd = 5{,}95$ ) i po tej interakcji ( $m = 22{,}02$ ; $sd = 5{,}93$ ) różnią się od siebie ( $T(299) = 45{,}62$ ; $p < 0{,}001$ ; $d = 0{,}52$ ) (Tab. \ref{tab:stress_12_t}). % $r_{pb} = 0{,}25$

\vspace{0.5cm}

\noindent
\textbf{Wniosek}, gdyby wynik nie był istotny statystycznie:

\vspace{0.2cm}

\noindent
Na poziomie istotności 0,05 nie możemy stwierdzić, czy pomiary stresu przed interakcją użytkownika z treścią ( $m = ...$ ; $sd = ...$ ) i po tej interakcji ( $m = ...$ ; $sd = ...$ ) różnią się od siebie ( $T(...) = ...$ ; $p = ...$ )  (Tab. \ref{tab:stress_12_t}).


\vspace{1cm}

\hrule

\vspace{0.5cm}


\subsubsection{Test Wilcoxona}

\vspace{0.5cm}

\hrule

\vspace{1cm}

\begin{center}
    \textbf{W rozdziale Metody:}
\end{center}

\vspace{0.5cm}

\noindent
\textbf{Pytanie badawcze}:\\
Czy pomiary stresu przed i po interakcji użytkownika z treścią różnią się od siebie? \\[0.6ex]
\textbf{Hipoteza badawcza}:\\
Pomiary stresu przed i po interakcji użytkownika z treścią różnią się od siebie.

\vspace{0.5cm}

[test wstępny, jeśli zmienna jest ilościowa]

\vspace{0.5cm}

\noindent
\textbf{Metoda}: test Wilcoxona

\vspace{1cm}

\hrule

\vspace{0.5cm}

\begin{center}
    \textbf{W rozdziale Wyniki:}
\end{center}

\vspace{0.5cm}

[wykres rozkładu łącznego zmiennych]

\vspace{0.5cm}

\noindent
\textbf{Hipotezy testowe}:

\vspace{0.2cm}

\noindent
$H_0:$ Mediana różnicy między pomiarami stresu przed i po interakcji użytkownika z treścią wynosi 0. \\[0.6ex]
$H_1:$ Mediana różnicy między pomiarami stresu przed i po interakcji użytkownika z treścią jest różna od 0.

\vspace{0.5cm}

\begin{table}[H]
\centering
\caption{Porównanie pomiarów stresu przed i po interakcji}
\label{tab:stress_12_w}
\vspace{0.35cm}
\tabelfont{
\begin{adjustbox}{center}
\begin{tabular}{l?ccc?>{\hspace{0.6em}}ccc<{\hspace{0.6em}}?}
    \cline{2-7}
  & \multicolumn{3}{c?}{\textbf{stres}} & \multicolumn{3}{c?}{\textbf{test Wilcoxona}} \\[0.6ex]
    \hline
\multicolumn{1}{|c?}{\textbf{pomiar}}   & $n$ & $med$ & $IQR$ & $W$  & $p$            & $CL$ \\[0.6ex]
    \thickhline
\multicolumn{1}{|c?}{przed interakcją}  & 300 & 25,00 & 11,00 & 0,00 & 0,00           & 0,64 \\
\multicolumn{1}{|c?}{po interakcji}     & 300 & 22,18 & 10,08 &      & $\ast\ast\ast$ &      \\[0.6ex]
    \hline
\end{tabular}
\end{adjustbox}}
{\raggedleft \scriptsize \hphantom{.} \\
%$\ast - p \leqslant 0{,}05$; \ $\ast \ast - p \leqslant 0{,}01$; \
$\ast \ast \ast - p \leqslant 0{,}001$}
\end{table}


\vspace{0.5cm}

\noindent
\textbf{Wniosek}:

\vspace{0.2cm}

\noindent
Pomiary stresu przed interakcją użytkownika z treścią ( $med = 25{,}00$ ; $IQR = 11{,}00$~) i~po tej interakcji ( $med = 22{,}18$ ; $IQR = 10{,}08$ ) różnią się od siebie ( $ W = 0{,}00$ ; $p < 0{,}001$~; $CL = 0{,}64$ )  (Tab. \ref{tab:stress_12_w}). % $r_{bc} = 1$ 

\vspace{0.5cm}

\noindent
\textbf{Wniosek}, gdyby wynik nie był istotny statystycznie:

\vspace{0.2cm}

\noindent
Na poziomie istotności 0,05 nie możemy stwierdzić, czy pomiary stresu przed interakcją użytkownika z treścią ( $med = ...$ ; $IQR = ...$ ) i po tej interakcji ( $med = ...$ ; $IQR = ...$~) różnią się od siebie ( $W = ...$ ; $p = ...$ )  (Tab. \ref{tab:stress_12_w}).


\vspace{1cm}

\hrule

\vspace{0.5cm}


%%%%%%%%%%%%%%%%%%%%%%%%%%%%%
\section{Testy siły związku}
%%%%%%%%%%%%%%%%%%%%%%%%%%%%%

\subsection{Test istotności współczynnika korelacji Pearsona}

\vspace{0.5cm}

\hrule

\vspace{1cm}

\begin{center}
    \textbf{W rozdziale Metody:}
\end{center}

\vspace{0.5cm}

\noindent
\textbf{Pytanie badawcze}:\\
Czy między wskaźnikami uwagi i zmienności nastroju istnieje związek? \\[0.6ex]
\textbf{Hipoteza badawcza}:\\
Między wskaźnikami uwagi i zmienności nastroju istnieje związek.

\vspace{0.5cm}

\noindent
\textbf{Metoda sprawdzająca założenie} współczynnika korelacji Pearsona 

\vspace{0.2cm}

\noindent
- normalność rozkładu: test Henze-Zirklera

\vspace{0.5cm}
\newpage
\noindent
\textbf{Wynik testów wstępnych}:

\vspace{0.2cm}

\noindent
Na poziomie istotności 0,05 nie mamy podstaw do odrzucenia hipotezy zerowej ( $HZ~=~0{,}58$~; $p = 0{,}63$ ), a więc przyjmiemy, że wskaźnik uwagi i wskaźnik zmienności nastroju w trakcie interakcji z treściami wystarczająco spełniają założenie normalności w teście istotności współczynnika korelacji Pearsona.

\vspace{0.5cm}

\noindent
\textbf{Metoda}: współczynnik korelacji Pearsona

\vspace{1cm}

\hrule

\vspace{0.5cm}

\begin{center}
    \textbf{W rozdziale Wyniki:}
\end{center}

\vspace{0.5cm}

[wykres rozkładu łącznego zmiennych]

\vspace{0.5cm}

\noindent
\textbf{Hipotezy testowe}:

\vspace{0.2cm}

\noindent
$H_0:$ Korelacja pomiędzy wskaźnikami uwagi i zmienności nastroju wynosi 0. \\[0.6ex]
$H_1:$ Korelacja pomiędzy wskaźnikami uwagi i zmienności nastroju jest różna od 0.

\vspace{0.5cm}


\begin{table}[H]
\centering
\caption{Statystyki opisowe i współczynnik korelacji dla wskaźników uwagi i zmienności nastroju}
\label{tab:attention_mood}
\vspace{0.35cm}
\tabelfont{
\begin{adjustbox}{center}
\begin{tabular}{l?ccccc?>{\hspace{2em}}ccc<{\hspace{2em}}?}
    \cline{2-9}
     & \multicolumn{5}{c?}{\textbf{rozkład zmiennej}} & \multicolumn{3}{c?}{\textbf{test korelacji Pearsona}}     \\[0.6ex]
    \hline
\multicolumn{1}{|l?}{\textbf{zmienna}} & $n$ & $m$ & $sd$ & $min$ & $max$ & $df$ & $r$ & $p$ \\[0.6ex]
    \thickhline
\multicolumn{1}{|l?}{wskaźnik uwagi}               & 300 & 0,60 & 0,14 & 0,12 & 1,00 & 298 & 0,11 & 0,052 \\
\multicolumn{1}{|l?}{wskaźnik zmienności nastroju} & 300 & 0,52 & 0,15 & 0,08 & 0,92 &     &      &      \\[0.6ex]
    \hline
\end{tabular}
\end{adjustbox}}
%{\raggedleft \scriptsize \hphantom{.} \\ $\ast - p \leqslant 0{,}05$; \ $\ast \ast - p \leqslant 0{,}01$; \ $\ast \ast \ast - p \leqslant 0{,}001$}
\end{table}


\vspace{0.5cm}

\noindent
\textbf{Wniosek}:

\vspace{0.2cm}

\noindent
Na poziomie istotności 0,05 nie można stwierdzić, czy między wskaźnikami uwagi i zmienności nastroju istnieje związek ( $ r(298) = 0{,}11$ ; $p = 0{,}052$ ) (Tab. \ref{tab:attention_mood}).

\vspace{0.5cm}

\noindent
\textbf{Wniosek}, gdyby wynik był istotny statystycznie:

\vspace{0.2cm}

\noindent
Między wskaźnikami uwagi i zmienności nastroju istnieje związek o zgodnym/przeciwnym kierunku ( $ r(...) = ...$ ; $p = ...$ ) - im wyższy wskaźnik uwagi, tym wyższy wskaźnik zmienności nastroju (Tab. \ref{tab:attention_mood}).


\vspace{1cm}

\hrule

\vspace{0.5cm}


\subsection{Test istotności współczynnika korelacji Spearmana}

\vspace{0.5cm}

\hrule

\vspace{1cm}

\begin{center}
    \textbf{W rozdziale Metody:}
\end{center}

\vspace{0.5cm}

\noindent
\textbf{Pytanie badawcze}:\\
Czy między trudnością treści i wskaźnikiem informacji zwrotnej istnieje związek? \\[0.6ex]
\textbf{Hipoteza badawcza}:\\
Między trudnością treści i wskaźnikiem informacji zwrotnej istnieje związek.

\vspace{0.5cm}

[test wstępny, jeśli zmienna jest ilościowa]

\vspace{0.5cm}

\noindent
\textbf{Metoda}: współczynnik korelacji Spearmana

\vspace{1cm}

\hrule

\vspace{0.5cm}

\begin{center}
    \textbf{W rozdziale Wyniki:}
\end{center}

\vspace{0.5cm}

[wykres rozkładu łącznego zmiennych]

\vspace{0.5cm}

\noindent
\textbf{Hipotezy testowe}:

\vspace{0.2cm}

\noindent
$H_0:$ Korelacja pomiędzy trudnością treści i wskaźnikiem informacji zwrotnej wynosi 0. \\[0.6ex]
$H_1:$ Korelacja pomiędzy trudnością treści i wskaźnikiem informacji zwrotnej jest różna od 0.

\vspace{0.5cm}


\begin{table}[H]
\centering
\caption{Statystyki opisowe i współczynnik korelacji dla trudności treści i wskaźnika informacji zwrotnej}
\label{tab:difficulty_feedback}
\vspace{0.35cm}
\tabelfont{
\begin{adjustbox}{center}
\begin{tabular}{l?ccccc?>{\hspace{3.5em}}cc<{\hspace{3.5em}}?}
    \cline{2-8}
     & \multicolumn{5}{c?}{\textbf{rozkład zmiennej}} & \multicolumn{2}{c?}{\textbf{test korelacji Spearmana}}     \\[0.6ex]
    \hline
\multicolumn{1}{|l?}{\textbf{zmienna}} & $n$ & $med$ & $IQR$ & $min$ & $max$ & $r_S$ & $p$ \\[0.6ex]
    \thickhline
\multicolumn{1}{|l?}{trudność treści}              & 300 & 2 & 2 & 1 & 3 & 0,14 & 0,014   \\
\multicolumn{1}{|l?}{wskaźnik informacji zwrotnej} & 300 & 3 & 2 & 1 & 5 &      & $\ast$ \\[0.6ex]
    \hline
\end{tabular}
\end{adjustbox}}
{\raggedleft \scriptsize \hphantom{.} \newline $\ast - p \leqslant 0{,}05$
%; \ $\ast \ast - p \leqslant 0{,}01$; \ $\ast \ast \ast - p \leqslant 0{,}001$
}
\end{table}


\vspace{0.5cm}

\noindent
\textbf{Wniosek}:

\vspace{0.2cm}

\noindent
Między trudnością treści i wskaźnikiem informacji zwrotnej istnieje związek o zgodnym kierunku ( $r_S = 0{,}14$ ; $p = 0{,}014$ ) - im większa trudność treści, tym wyższy wskaźnik informacji zwrotnej (Tab. \ref{tab:difficulty_feedback}).

\vspace{0.5cm}

\noindent
\textbf{Wniosek}, gdyby wynik nie był istotny statystycznie:

\vspace{0.2cm}

\noindent
Na poziomie istotności 0,05 nie można stwierdzić, czy między trudnością treści i wskaźnikiem informacji zwrotnej istnieje związek ( $ r_S(...) = ...$ ; $p = ...$ ) (Tab. \ref{tab:difficulty_feedback}).

\vspace{1cm}

\hrule

\vspace{0.5cm}

\subsection{Test niezależności zmiennych \texorpdfstring{$\chi^2$}{TEXT}}

\vspace{0.5cm}

\hrule

\vspace{1cm}

\begin{center}
    \textbf{W rozdziale Metody:}
\end{center}

\vspace{0.5cm}

\noindent
\textbf{Pytanie badawcze}: Czy między płcią i zawodem istnieje związek? \\[0.6ex]
\textbf{Hipoteza badawcza}: Między płcią i zawodem istnieje związek.

\vspace{0.5cm}

\noindent
\textbf{Metoda}: test niezależności $\chi^2$ ze współczynnikiem korelacji $V$ Craméra

\vspace{1cm}

\hrule

\vspace{0.5cm}

\begin{center}
    \textbf{W rozdziale Wyniki:}
\end{center}

\vspace{0.5cm}

[wykres rozkładu łącznego zmiennych]

\vspace{0.5cm}

\noindent
\textbf{Hipotezy testowe}:

\vspace{0.2cm}

\noindent
$H_0:$ Płeć i zawód są od siebie niezależne. \\[0.6ex]
$H_1:$ Płeć i zawód są od siebie zależne.

\vspace{0.5cm}



\begin{table}[H]
\centering
\caption{Tabela krzyżowa i test niezależności $\chi^2$ dla płci i zawodu}
\label{tab:gender_occupation}
\vspace{0.35cm}
\tabelfont{
\begin{adjustbox}{center}
\begin{tabular}{l?ccccc?>{\hspace{0.02em}}cccc<{\hspace{0.02em}}?}
    \cline{2-10}
     & \multicolumn{5}{c?}{\textbf{zawód} ($n$)} & \multicolumn{4}{c?}{\textbf{test niezależności $\chi^2$}}     \\[0.6ex]
    \hline
\multicolumn{1}{|l?}{\textbf{płeć}} & inżynier & lekarz & uczeń & nauczyciel & bezrobotny & $df$ & $\chi^2$ & $p$ & $V$ \\[0.6ex]
    \thickhline
\multicolumn{1}{|l?}{kobieta}       & 2 & 4 & 6 & 6 & 3 & 4    & 3,43 & 0,49 & 0,26 \\
\multicolumn{1}{|l?}{mężczyzna}     & 8 & 3 & 5 & 8 & 5 &      &      &      &      \\[0.6ex]
    \hline
\end{tabular}
\end{adjustbox}}
%{\raggedleft \scriptsize \hphantom{.} \\ $\ast - p \leqslant 0{,}05$; \ $\ast \ast - p \leqslant 0{,}01$; \ $\ast \ast \ast - p \leqslant 0{,}001$}
\end{table}

\indent [jeśli tabela krzyżowa ma wiele pól, można przedstawić ją osobno]


\vspace{0.5cm}
\newpage
\noindent
\textbf{Wniosek}:

\vspace{0.2cm}

\noindent
Na poziomie istotności 0,05 nie można stwierdzić, czy między płcią i zawodem, istnieje związek ( $\chi^2(4) = 3{,}43$ ; $p = 0{,}49$ ; $V = 0{,}26$ ) (Tab. \ref{tab:gender_occupation}).

\vspace{0.5cm}

\noindent
\textbf{Wniosek}, gdyby wynik był istotny statystycznie:

\vspace{0.2cm}

\noindent
Między płcią i zawodem istnieje związek ( $\chi^2(...) = ...$ ; $p = ...$ ; $V = ...$ ) (Tab. \ref{tab:gender_occupation}). Największa zależność występuje w zawodzie inżyniera - mężczyźni częściej zostają inżynierami niż kobiety/pozostali przedstawiciele płci [gdyby było więcej poziomów].

\vspace{0.5cm}

\noindent
\textbf{Wniosek}, gdyby wynik był istotny statystycznie i tabela krzyżowa była wymiaru $2 \times 2$:

\vspace{0.2cm}

\noindent
Między płcią i byciem inżynierem istnieje związek ( $\chi^2(...) = ...$ ; $p = ...$ ; $V = ...$ ) - mężczyźni częściej zostają inżynierami niż kobiety (Tab. \ref{tab:gender_occupation}).

\vspace{1cm}

\hrule

\vspace{0.5cm}










\end{document}


